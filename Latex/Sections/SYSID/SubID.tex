\subsection{Subspace identification using general input sequences}
Subspace identification methods can be used to retrieve subspaces that are related to the system matrices of the signal generating state-space model \ref{eq:sgs_x},\ref{eq:sgs_y}. This is done by storing the input and output data in structured block Hankel matrices. For the heater system the "Multivariable Output-Error State-Space" (MOESP) method \cite[p.~301--312]{FilteringIdentification} of subspace identification will be used to calculate the column space of the extended observability matrix $\mathcal{O}_s$. Out of which the system matrices will be determined up to a certain similarity transformation. 

\begin{align}
    x(k+1) &= Ax(k) + Bu(k) \label{eq:sgs_x}\\
    y(k) &= Cx(k) + Du(k) + v(k) \label{eq:sgs_y}
\end{align}

\subsubsection{MOESP}
The MOESP method uses RQ factorization to find the matrix $R_{22}$ which has the same column space as the extended observability matrix $\mathcal{O}_s$. 
$$
\left[\begin{array}{c}
U_{0, s, N} \\
Y_{0, s, N}
\end{array}\right]=\left[\begin{array}{ccc}
R_{11} & 0 & 0 \\
R_{21} & R_{22} & 0
\end{array}\right]\left[\begin{array}{c}
Q_{1} \\
Q_{2} \\
Q_{3}
\end{array}\right]
$$
where $U_{0,s,N}$ and $Y_{0,s,N}$ are block Hankel matrices constructed from the inputs and outputs respectively which are definded as follows
$$
H_{i, s, N}=\left[\begin{array}{cccc}
h(i) & h(i+1) & \dots & h(i+N-1) \\
h(i+1) & h(i+2) & \dots & h(i+N) \\
\vdots & \vdots & \ddots & \vdots \\
h(i+s-1) & h(i+s) & \dots & h(i+N+s-2)
\end{array}\right]
$$


\subsubsection{Model validation}
ziet er mooi uit man