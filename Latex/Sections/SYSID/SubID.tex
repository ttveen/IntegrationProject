\subsection{Subspace identification using general input sequences}
Subspace identification is a system identification method that can be used to retrieve certain subspaces, from the input and output data structured in block Hankel matricesthat, that are related to the system matrices of the signal generating state-space model (eq: \ref{eq:sgs_x},\ref{eq:sgs_y}). For the heater system the "Multivariable Output-Error State-Space" (MOESP) method \cite[p.~301--312]{FilteringIdentification} of subspace identification will be used to calculate the column space of the extended observability matrix $\mathcal{O}_s$ (eq: \ref{eq:extendedObservability}). Out of which the system matrices will be determined up to a certain similarity transformation. 

\begin{align}
    x(k+1) &= Ax(k) + Bu(k) \label{eq:sgs_x}\\
    y(k) &= Cx(k) + Du(k) + v(k) \label{eq:sgs_y}
\end{align}

\begin{equation}
\mathcal{O}_s = \left[\begin{array}{c}
C \\
C A \\
C A^{2} \\
\vdots \\
C A^{s-1}
\end{array}\right]
\label{eq:extendedObservability}
\end{equation}

\subsubsection{MOESP}
The MOESP method uses RQ factorization to find the matrix $R_{22}$ \ref{eq:RQ} which has the same column space as the extended observability matrix $\mathcal{O}_s$. Prove of which can be found in \cite[p.~304--305]{FilteringIdentification}.
\begin{equation}
    \left[\begin{array}{c}
        U_{0, s, N} \\
        Y_{0, s, N}
    \end{array}\right]=\left[\begin{array}{ccc}
        R_{11} & 0 & 0 \\
        R_{21} & R_{22} & 0
    \end{array}\right]\left[\begin{array}{c}
        Q_{1} \\
        Q_{2} \\
        Q_{3}
    \end{array}\right]
    \label{eq:RQ}
\end{equation}
where $U_{0,s,N}$ and $Y_{0,s,N}$ are block Hankel matrices constructed from the inputs and outputs respectively which are defined as follows
$$
H_{i, s, N}=\left[\begin{array}{cccc}
h(i) & h(i+1) & \dots & h(i+N-1) \\
h(i+1) & h(i+2) & \dots & h(i+N) \\
\vdots & \vdots & \ddots & \vdots \\
h(i+s-1) & h(i+s) & \dots & h(i+N+s-2)
\end{array}\right]
$$
The system matrices $A_T$ and $C_T$ can be determined by taking the SVD of the matrix $R_{22}$.
$$
R_{22} = U_n\Sigma_nV_n^T
$$
where n is the rank of the system which can be estimated by determining the amount of distinct singular values of $Y_{0,s,N}$ (fig: \ref{fig:SingularY0sN}). 
% which was found to be $n\approx7$.  
$C_T$ then equals the first $l$ rows of $U_n$
$$
C_t = U_n(1:l,:)
$$ 
where l is the amount of outputs of the system.
$A_T$ is determined by solving the following equation
$$
U_n(1:(s-1)l,:)A_T = U_n(l+1:sl,:)
$$
To determine the matrices $B_T$ and $D_T$, the following least-squares problem needs to be solved for $\theta$
$$
    \min _{\theta} \frac{1}{N} \sum_{k=0}^{N-1}\left\|y(k)-\phi(k)^{\mathrm{T}} \theta\right\|_{2}^{2}
$$
where
$$
    \phi(k)^{\mathrm{T}}=\left[\widehat{C}_{T} \widehat{A}_{T}^{k} \quad\left(\sum_{\tau=0}^{k-1} u(\tau)^{\mathrm{T}} \otimes \widehat{C}_{T} \widehat{A}_{T}^{k-\tau-1}\right) \quad\left(u(k)^{\mathrm{T}} \otimes I_{\ell}\right)\right]
$$
and
$$
    \theta=\left[\begin{array}{c}
    x_{T}(0) \\
    \operatorname{vec}\left(B_{T}\right) \\
    \operatorname{vec}\left(D_{T}\right)
    \end{array}\right]
$$
From the vector $\theta$ the matrices $B_T$ and $D_T$ can be constructed. $\theta$ also contains the transformed initial states.

\subsubsection{Results}
From figure \ref{fig:SingularY0sN} it is determined, by the amount of distinct singular values of $Y_{0,s,N}$, that the system has $n \approx 9$ states when the Hankel matrices have been created with $n < s = 150 < N$.
\begin{figure}[ht]
    \centering
    \includesvg[width=0.8\textwidth]{images/SYSID/singularVal.svg}
    \caption{Singular values of Y0sN}
    \label{fig:SingularY0sN}
\end{figure}
\begin{figure}[ht]
    \centering
    \includesvg[width=0.8\textwidth]{images/SYSID/compSubID.svg}
    \caption{Comparison between the outputs of the experiment and the system created with subspace identification}
    \label{fig:Comp_subID}
\end{figure}
From figure \ref{fig:Comp_subID} it can be seen that the model determined with MOESP, given the same input, mostly coincides with the experiment data. It is even able to match well with the output of Heater 2 which has, as seen in the figure, more measurement noise than Heater 1.

% The relative error between the output of the model determined with subspace identification and the experiment data can be calculated with
% $$
% Relative error(k) = \frac{|y_{exp}-y_{subID}|}{|y_{exp}|}
% $$
