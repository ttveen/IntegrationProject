\subsection{Choice of sampling frequency}

\begin{figure}[ht]
    \centering
    \includesvg[width=0.8\textwidth]{images/SYSID/stepResponse.svg}
    \caption{Open-loop step response of the system with an input of 50\%}
    \label{fig:stepResponse1}
\end{figure}

A good choice of the sampling frequency is necessary to fully capture the system's output.
A rule of thump is to select the sampling frequency 
$$
\omega_S = 10\omega_B
$$
where $\omega_B$ is the bandwidth of the system. \\

\noindent The system bandwidth can be estimated by using the following relationship between the bandwidth and rise time of the system's step response.
$$
 T = \frac{2\pi}{10\omega_B}
$$
where T is the interval between about eight sampled points on the rise time which is approximately sampling at $10\omega_B$. $\omega_B$ can thus be estimated by rewriting the above relationship and substituting $T$. \\

\noindent The results for both heaters can be found in table \ref{Tab:stepInfo}. These values have been derived from Figure \ref{fig:stepResponse1}.

\begin{table}[ht]
\centering
\caption{Open-loop step response information}
\begin{tabular}{lllll}
 & Steady State ($^{\circ}C$) & Rise Time (s) & $\omega_B$ (rad/s) & $\omega_S$ (rad/s) \\ \cline{2-5} 
Heater 1 & 67.79 & 318 & 0.016 & 0.16 \\
Heater 2 & 57.04 & 357 & 0.014 & 0.14
\end{tabular}
\label{Tab:stepInfo}
\end{table}
From these results it can be seen that taking a measurement every $T_{s,1} = \frac{2\pi}{\omega_{s,1}} \approx 39$ seconds for Heater 1 and $T_{s,2} = \frac{2\pi}{\omega_{s,2}} \approx 45$ seconds for Heater 2 should result in the system's output to be fully captured. In the following system identification methods a sampling time of 1 second has been used to record the experiment data. The sampling time can then be decreased in a later stage if necessary by sampling the recorded data itself. 