\subsection{System identification comparison}
In this chapter two system identification methods were used to obtain a model. From the model validation it was made apparent that the model obtained by subspace identification represented the real system far more accurately than the least squares fit, thus it is not necessary to compute an error of both models. These errors could be used to make a definitive conclusion about the performance of the models, so that a better model could be chosen.\\
Since the performance of the identified models differ so much, it is easy to conclude that subspace identification is superior to the least squares approach. The least squares approach does offer some benefits that the subspace identification lacks.
\begin{itemize}
    \item The linear least squares offers a model where meaning of the states is known. In the case of the system inspected in this report, the states represent the temperature of the heaters and sensors. Subspace identification returns a model of order $n$. The meaning of the $n$ states is lost, since subspace identification utilises the rank and column space of the input and output data. The obtained model could be any linear transformation, since a linear transformation does not change the rank or column space.
    \item The linear least squares approach offers a non-linear model. Subspace identification results in a linear model. One is not necessarily better than the other, since linearising the nonlinear model might result in inaccuracies, further away from the linearisation point. However, a nonlinear model could capture dynamics that a linear model could not. These dynamics can then be controlled and observed by nonlinear controllers and observers.
\end{itemize}
Note that the subspace identification also uses linear least sqaures to minimise an error. Some linear least squares is thus used, whatever method is chosen.

One of the possibilities of the subspace method, is to estimate the covariance of the measurement and process noise. This will be exploited in the next chapter, where a kalman filter is designed.