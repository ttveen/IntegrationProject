\subsection{Theoretical system model}\label{TSM}
For system identification, it is useful to know the order of system. A theoretical model can be used as basis for obtaining the order. The dynamics of the two heaters and temperature sensors can be described by \cite{APMonitor}
\begin{align}
    mc_p\dot{T}_1 &= U A (T_{\infty} - T_1) + \epsilon\sigma(T^4_{\infty} - T1^4_1) + Q_{C12} + Q_{R12} +\alpha_1 Q_1 \nonumber \\
    mc_p\dot{T}_2 &= U A (T_{\infty} - T_2) + \epsilon\sigma(T^4_{\infty} - T1^4_2) - Q_{C12} - Q_{R12} +\alpha_2 Q_2 \nonumber
\end{align}

The parameters can be found in table \ref{tab:theModelPara}. To obtain these equations, the following derivation can be used:
$$
    mc_p\dot{T} = \sum\dot{h}_{in} - \sum\dot{h}_{out} + Q
$$
where $\dot{h}$ represents the change in enthalpy. The energy leaving the system, $ \sum\dot{h}_{out}$, consist of the convective term  $U A (T_{\infty} - T)$ and the radiation term $\epsilon\sigma(T^4_{\infty} - T1^4_1)$ with the ambient environment. The incoming energy consist of the energy flowing from the other heater, convective and radiation
\begin{align}
    Q_{C12} &= U A_s (T_2 -T_1) \nonumber \\
    Q_{R12} &= \epsilon\sigma(T^4_{2} - T1^4_1) \nonumber
\end{align}
Of course, the heat transfer from heater 1 to heater 2 is outgoing for heater 1, and incoming for heater 2. $Q_1$ and $Q_2$ represent the input heat, generated by the transistors. These inputs are scaled by a factor $\alpha$.
\begin{table}[ht]
    \centering
    \begin{tabular}{l|l}
    \textbf{Quantity} & \textbf{Value} \\
    Ambient temperature $T_{\infty}$    & \\
    Heater output $Q_i$ & $0<Q_1<1$, $0<Q_2<0.75$\\
    Heater factor $\alpha_i$ & $\alpha_1 = 0.01W/\%$, $\alpha_2 = 0.075W/\%$ \\
    Heat capacity $C_p$ & 500 J/(kg K)\\
    Surface Area between heaters $A_s$ & $2.0\times10^{-4} m^2$\\
    Surface Area not between heaters $A$ & $1.0\times10^{-3} m^2$\\
    Mass $m$ & 0.004 kg\\
    Heat Transfer Coefficient $U$ & $10$ W/($m^2$ K) \\
    Emissivity $\epsilon$ & 0.9\\
    Boltzmann Constant $\sigma$ & $5.67\times10^{-8}$ W/($m^2 K^4$) \\
    Mass sensor $m_s$ & \\
    Heat capacity Sensor $C_{p,s}$ & \\
    \end{tabular}
    \caption{The parameters of the model, provided by \cite{APMonitor}}
    \label{tab:theModelPara}
\end{table}
% The dynamics of the temperature sensors can be described by
The temperature sensors also have their own dynamics which can be described by
\begin{align*}
    \tau \frac{d T_{C 1}}{d t} &= T_{H 1}-T_{C_{1}} \\ 
    \tau \frac{d T_{C 2}}{d t} &= T_{H 2}-T_{C_{2}}
\end{align*}
where $\tau$ is the conduction time constant which is equal to 
$$
\tau = \frac{m_{s} c_{p, s} \Delta x}{k A_{c}}
$$
These differential equations are derived from the rate of heat flow equation below
$$
\frac{\Delta Q}{\Delta t}=-k A \frac{\Delta T}{\Delta x}
$$
where \\
$\Delta Q$, is the net heat transfer\\
$k$, is the thermal conductivity\\
$A$, is the surface area the heat emitting surface\\
$\Delta x$, is the thickness of the material conducting the heat.