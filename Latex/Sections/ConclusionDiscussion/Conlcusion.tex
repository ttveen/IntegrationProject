In this report, successful methods for system identification and controller design for the Temperature Control Lab have been described.\\
Two methods for system identifications have been considered, a linear least square parameter fit based on a theoretical model, and a subspace method, MOESP. The MOESP greatly outperforms the parameter fit, using the RMS error as comparison. The MOESP method allows for estimation of the system order, and provided a linear sixth order model that was able to accurately model the measured outputs, given the same inputs.\\
Subspace identification also allowed for the estimation of the covariance of the process- and measurement noise using the N4SID method. These estimation allowed for the design of a Kalman filter. The model found by the N4SID method is, with addition of the Kalman filter, the most accurate model found.\\
The identified N4SID model is used as a basis for two different control approaches. A pole-placement controller, and a MPC controller were considered. Both controllers were used to track a reference signal, both a step reference and a sinusoidal reference. The pole-placement controller was able to decently track the step reference, but not the sinusoidal reference. The MPC controller outperformed the pole-placement controller on all tests. It was able to follow sinusoidal references of lower frequencies. Higher frequencies references resulted in over- and undershoot. Additionally, the MPC controller allows for hard constraints on the inputs, which decreases the risk of overheating the Temperature Control Lab.