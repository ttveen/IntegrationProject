To check if the dynamical system is linear the following linearity condition needs to apply
$$
f(t,x,au_a+bu_b) = af(t,x,u_a) + bf(t,x,u_b)
$$
By choosing the values 
$$
a = b = 1 \text{ and } u_a = u_b = 0.25
$$
Two different simulations are obtained
$$
f(t,x,0.5) = 2f(t,x,0.25)
$$
that need to be performed and compared to each other to check if the above linearity condition apply.
Figure \ref{fig:LinComp} shows the comparison between the two performed simulations.

\begin{figure}[ht]
    \centering
    \includesvg[width=0.8\textwidth]{images/Linearity/Comparison.svg}
    \caption{Input linearity comparison of the two heaters}
    \label{fig:LinComp}
\end{figure}

From figure \ref{fig:LinComp} it can be seen that there is a slight difference between the two simulations but as the outputs are still running closely parallel to each other it can be assumed that the system is linear and that the difference can be attributed to other sources such as the initial states and disturbances.\\ 

A second check is done by comparing the output, when the two inputs are applied separately. In a linear system the response of the sum of the seperate inputs would be equal to the response of the system with the inputs applied simultaneously.
$$
f(t,x,[0.5, \ 0.5]^T) = f(t,x,[0.5, \ 0]^T) + f(t,x,[0, \ 0.5]^T)
$$
\begin{figure}[ht]
    \centering
    \includesvg[width=0.8\textwidth]{images/Linearity/ComparisonSeperate.svg}
    \caption{Input linearity comparison of two heaters running separately}
    \label{fig:linComp2}
\end{figure}

Figure \ref{fig:linComp2} shows the comparison. The red and blue dots represent the temperature during the experiment with both heaters on 50\%. The green and black dots are the summed measurements of the experiment where first only heater 1 was on 50\%, and then only heater 2 on 50\%. The temperatures of the experiment two are slightly lower than in the first experiment, similar to the first linearity check. \\

Both checks indicate that the system in non-linear, but the differences are rather small. A linearised model is expected to perform satisfactory around the linearisation point. In the next chapter two different system identifcation methods will be used to find a linear system model.

