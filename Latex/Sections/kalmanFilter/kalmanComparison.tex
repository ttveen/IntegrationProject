\subsection{Kalman filter comparison}
From figures \ref{fig:comp_Kalman1} and \ref{fig:comp_Kalman2} it can be seen that the output estimated with the Kalman filter slowly converges to the measured output. The estimated output of heater 1 converges quicker than that of heater 2. After 30 seconds, both estimates are near the measurements. The duration of the convergence may appear slow, but it corresponds to the slower dynamics of the system. During the first 30 seconds of an experiment or measurement, the estimate is significantly off, thus the controller will give inputs to the system that are significantly off too. Once the estimate is very similar to the measurement, the estimate and measurement evolve similar over time. This is expected behaviour for a linear observer or Kalman filter.

Note that the initial estimates were intentionally set to $T_{C1,measured} = T_{C1,measured} = 0$. There is a significant difference between the initial estimates and the initial temperature of the sensors (room temperature). Choosing this difference allows for the inspection of the convergence time. Performance will increase when the initial conditions of the estimate are closer to the real conditions.

The RMS error for both experiments, with and without convergance, are
The RMS of the N4SID model is
\begin{align*}
\epsilon_{\text{KF, experiment 1}} = \begin{bmatrix} 3.6262 & 4.1523 \end{bmatrix}\\
\epsilon_{\text{KF, eperiment 2}} = \begin{bmatrix} 3.0735 & 4.0106 \end{bmatrix}\\
\epsilon_{\text{KF, } t\geq40,\text{experiment 1}} = \begin{bmatrix} 0.7861 & 1.5044 \end{bmatrix}\\
\epsilon_{\text{KF, } t\geq40, \text{experiment 2}} = \begin{bmatrix} 0.6666 & 1.4358 \end{bmatrix}\\
\end{align*}
These errors are higher on the second output than the first. For the second output, the RMS errors of the two experiments are significantly higher than the RMS error on the original fitting data. This might indicate that the model was over fitting on the noise of the second output. This is plausible, since the second output appears to more noisy than the first output.
\begin{figure}
    \centering
    \includesvg[width=\textwidth]{images/kalmanTest/kalmanTest1.svg}
    \caption{Comparison of the measured output and the estimated output, experiment 1}
    \label{fig:comp_Kalman1}
\end{figure}
\begin{figure}
    \centering
    \includesvg[width=\textwidth]{images/kalmanTest/kalmanTest2.svg}
    \caption{Comparison of the measured output and the estimated output, experiment 2}
    \label{fig:comp_Kalman2}
\end{figure}
